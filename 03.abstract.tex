\noindent
Energy-efficiency has concerned hardware and low-level software designers for years. However, the rapid proliferation of battery-powered mobile devices combined with the growing worldwide movement towards sustainability have caused developers and researchers to study the energy impact of application software in execution. Recent work has studied the effect that factors such as code obfuscation, object-oriented refactorings, and data types have on energy efficiency. In this work, we attempt to shed light on the energy behavior of concurrent programs written in a purely functional language, Haskell. %
We conducted an empirical study to assess the performance and energy behavior of three different thread management approaches and three primitives for concurrency control using nine different benchmarks with an experimental space exploration of more than 400 configurations. In this study, we found out that small changes can make a big difference in terms of energy consumption. For instance, in one of our benchmarks, under a specific configuration, choosing one concurrency control primitive (MVar) over another (TMVar) can yield 60\% energy savings. Also, the relationship between energy consumption and performance is not always clear. We found scenarios where the configuration with the best performance also exhibited the worst energy consumption. %
To support developers in better understanding this complex relationship, we have extended two existing performance analysis tools also to collect and present data about energy consumption. In addition, based on the results of our empirical study, we provide a list of guidelines for developers with good practices for writing energy-efficient code in this environment.

\begin{keywords}
Energy-Efficiency. Energy Consumption. Haskell. Concurrent Programming. Functional Programming. Performance Analysis.
\end{keywords}
