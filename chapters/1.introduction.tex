\chapter{Introduction}\label{chp:introduction}

% \begin{quotation}[]{Poul Anderson}
% I have yet to see any problem, however complicated, which, when looked at in the
% right way, did not become still more complicated.
% \end{quotation}

The evolution of technology is unveiling a reality that could hardly be imaginable some decades ago. Nowadays, we can buy a multiprocessor computer with a few gigabytes of memory with the size of a credit card for less than a hundred dollars. The ability to manufacture such small and potent devices is leading to a rapid proliferation of a variety of mobile computing platforms. These devices are part of a diverse ecosystem that includes smartwatches, smartphones, tablets, IoT sensors, and drones. In this sort of device, it is imperative for delivering a good user experience that they stay up and running for as long as possible. So \emph{energy consumption} is a huge concern in this universe as it is closely related to battery lifetime. %% XXX: the conclusion still need some love

This concern, however, goes beyond unwired devices. On the other side of the spectrum, big internet companies are also affected by low energy efficiency. To deliver fast access to its services globally, these companies usually have to maintain a huge server infrastructure to host its products. In this kind of environment, due to its scale, the energy consumption can have a high impact on the maintenance costs. For instance, Facebook is building data centers inside the Arctic Circle in order to improve the system's energy efficiency by reducing the power needed for cooling\footnote{\scriptsize\url{http://www.bloomberg.com/news/articles/2013-10-04/facebooks-new-data-center-in-sweden-puts-the-heat-on-hardware-makers}}. This is just one example of the economic impact driven by the seek for more efficient solutions. It shows that energy efficiency is becoming a key design attribute for building computational systems.

Although it may seem like a recent problem, the energy efficiency of computer systems has been a concern for researchers for a long time. Initially, most of the research focused on the hardware design layer, developing new ways to build electronic components that wasted less energy~\cite{chandrakasan:1992}. This was motivated by the assumption that only hardware dissipates power, not software. However, in a computer system, software plays a fundamental role in deciding how a computational task will be executed on specific hardware. For this reason, software can have a substantial impact on energy consumption.

From a software perspective, the energy efficiency problem can be tackled at different levels of abstraction, ranging from machine code level to user-facing applications. Traditionally, the research in this area has been focused on low-level software. Much progress has been achieved on building energy-efficient solutions for embedded software~\cite{tiwari:1994}, compilers~\cite{hsu:2003}, operating systems~\cite{merkel:2006} and runtime systems~\cite{ribic:2014, farkas:2000}. However, the growing worldwide movement towards sustainability, including sustainability in software~\cite{becker:2015}, have motivated the study of the energy impact of application software in execution.

Recent empirical studies have provided initial evidence that high-level decisions can effectively reduce the energy usage of an application software~\cite{chung:2001,hindle:2012,pinto:2014,trefethen:2013,manotas:2014,sahin:2014}. It is important to note that optimizing software at the application level does not cancel the lower level optimizations. They are complementary solutions. However, it is still not clear which software engineering practices are beneficial for saving energy. A recent study by~\citeonline{pinto:2014b} shows that, although application developers are consistently more interested in understanding how to reduce energy consumption in their software, there is a general lack of information in the community about how it can be achieved.

In this work, we tackle the software energy consumption problem at the application level. Moreover, we study energy consumption in the context of Concurrent Haskell. We aim to help Haskell developers to better understand how its decisions can impact the energy consumption and performance when developing  concurrent sofware.


\section{Problem}
\lipsum[1-2]


\section{Goal}
\lipsum[3-4]


\section{Contributions}
\lipsum[7-8]

\section{Outline}
The remainder of this work is structured as follows.

\textbf{\chapref{chp:background}} reviews essential concepts used throughout this work. First, we briefly introduce the Haskell programming language. We show through a series of code samples some important and distinct features of the language. Second, we present an overview of the fundamentals of concurrent programming. Finally, we present how Haskell approaches concurrent programming. We show which abstractions the language provides for both creating new threads of execution and sharing data among these threads. We also present an overview of how the Haskell's runtime system handles threading on multiprocessors.

In \textbf{\chapref{chp:tools}}, we show how to measure the energy consumption of Haskell programs. First, we explain what is \acs{rapl} and how it can be used to collect energy data. Then, we present in details two performance analysis tools of the Haskell ecosystem that we extended to also work with the energy consumption metric.

\textbf{\chapref{chp:study}} shows how different concurrent constructs impact energy consumption. We present an empirical study that we conducted considering three distinct thread management constructs and primitives for sharing data. Through an extensive experimental space exploration over real-world Haskell benchmarks, we produce a list of findings about the energy behaviors of concurrent Haskell programs, which are not always obvious.

In \textbf{\chapref{chp:guide}}, we present a list of recommendations for Haskell developers on how to write energy-efficient software. These recommendations are based on the results of our empirical study.

Finally, \textbf{\chapref{chp:related}} discusses previous research related to this work and \textbf{\chapref{chp:conclusion}} present our concluding remarks and discuss where this work might lead.
