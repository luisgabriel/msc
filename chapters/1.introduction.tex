\chapter{Introduction}\label{chp:introduction}

% \begin{quotation}[]{Poul Anderson}
% I have yet to see any problem, however complicated, which, when looked at in the
% right way, did not become still more complicated.
% \end{quotation}

The evolution of technology is unveiling a reality that could hardly be imaginable some decades ago. Nowadays, we can buy a multiprocessor computer with a few gigabytes of memory with the size of a credit card for less than a hundred dollars. The ability to manufacture such small and potent devices is leading to a rapid proliferation of a variety of mobile computing platforms. These devices are part of a diverse ecosystem that includes smartwatches, smartphones, tablets, IoT sensors, and drones. As they grow in popularity, new challenges arise for the development community. \emph{Energy consumption} is one of them. It is imperative for delivering a good user experience that these devices stay up and running for as long as possible. As battery lifetime is closely related to energy consumption, it means that building energy-efficient systems is becoming mandatory for providing value to the end user.

This concern, however, goes beyond unwired devices. On the other side of the spectrum, big internet companies are also affected by low energy efficiency. To deliver fast access to its services globally, these companies usually have to maintain a huge server infrastructure to host its products. In this kind of environment, due to its scale, the energy consumption can have a high impact on the maintenance costs. For instance, Facebook is building data centers inside the Arctic Circle in order to improve the system's energy efficiency by reducing the power needed for cooling\footnote{\scriptsize\url{http://www.bloomberg.com/news/articles/2013-10-04/facebooks-new-data-center-in-sweden-puts-the-heat-on-hardware-makers}}. This is just one example of the economic impact driven by the seek for more efficient solutions. It shows that energy efficiency is becoming a key design attribute for building computational systems.

Although it may seem like a recent problem, the energy efficiency of computer systems has been a concern for researchers for a long time. Initially, most of the research focused on the hardware design layer, developing new ways to build electronic components that wasted less energy~\cite{chandrakasan:1992}. This was motivated by the assumption that only hardware dissipates power, not software. However, in a computer system, software plays a fundamental role in deciding how a computational task will be executed on specific hardware. For this reason, software can have a substantial impact on energy consumption.

From a software perspective, the energy efficiency problem can be tackled at different levels of abstraction, ranging from machine code level to user-facing applications. Traditionally, the research in this area has been focused on low-level software. Much progress has been achieved on building energy-efficient solutions for embedded software~\cite{tiwari:1994}, compilers~\cite{hsu:2003}, operating systems~\cite{merkel:2006} and runtime systems~\cite{ribic:2014, farkas:2000}. However, the growing worldwide movement towards sustainability, including sustainability in software~\cite{becker:2015}, have motivated the study of the energy impact of application software in execution.

Recent empirical studies have provided initial evidence that high-level decisions can effectively reduce the energy usage of an application software~\cite{hindle:2012,trefethen:2013,pinto:2014,sahin:2014}. A big advantage of this kind of optimization is that they are complementary to the low-level ones, which helps improving the energy efficiency of the system as a whole. Also, it includes the developer in the loop of deciding how to optimize its software for a certain context. As they have the knowledge about the application domain, it can lead to more aggressive optimizations. In contrast, low-level optimizations have to be usually more generic and, consequently, more conservative.

Nevertheless, the programming models used for developing software is an important piece of this puzzle when we talk about writing energy-efficient software. A trend that we can identify in modern software development is the employment of concurrency techniques as a way improve the program's performance. The main reason for this is the increasing popularity of multicore processors. To leverage this kind of architecture, developers often have to create multiple flows of control in their programs and manually orchestrate them to guarantee correctness. This is, however, a difficult and error-prone task to accomplish~\cite{sutter:2005,herlihy:2012}.

A commonly referred alternative to mitigate this problem is the use of functional programming. Several functional programming languages such as Clojure, Erlang, Elixir, and Scala have gained popularity in the last decade emphasizing features that supposedly makes it easier dealing with concurrency. Features such as immutable data structures, Actor Model, and \acl{stm} are a few of them. Haskell is a language that stands out in this context for having many different abstractions for parallel and concurrent programming.

Finally, the relationship between energy consumption and concurrency is still uncertain. We have a vast literature on how to improve the performance of programs by employing concurrency, but we still do not understand with the same depth how they impact energy consumption. Depending on the scenario, the gains in performance may not worth the decrease in energy efficiency. Recent studies have shed light on this matter for the concurrency toolkit of the Java programming language~\cite{pinto:2014,pinto:2016}. In the functional programming world, however, this is a subject yet to be explored.

In this work, we believe that functional programming can play a major role in helping developers to write correct concurrent programs. Moreover, we think that educating developers and providing the necessary tools for them to write energy-efficient code in this environment is crucial for making sure they develop software that meets the usability and business requirements. To achieve this goal, we need to understand how high-level decisions change the behavior of a program regarding its performance and energy consumption.


\section{Problem}
In this work, we tackle two critical problems in the development of energy-efficient concurrent applications. The first one is the \emph{lack of knowledge}. A recent study by~\citeonline{pinto:2014b} shows that, although application developers are consistently more interested in understanding how to reduce energy consumption in their software, there is a general lack of information in the community about how it can be achieved. This is unfortunate because it is very unlikely that developers will be able to build energy-efficient solutions without a logical and systematical way of reasoning about the energy consumption of the software that they write.

The second problem is the \emph{lack of tools}. Currently, there is not much tooling in which developers can rely on to analyze the energy consumption of their programs. Most of the tools available are more close to the systems side than the software side, which makes difficult to establish a relation with the software in execution. This problem is closely related to the previous one as the availability of such tools can lead to a better understanding about software energy consumption in general.


\section{Goal}
The goal of this work is to mitigate both problems aforementioned: the lack of knowledge, and the lack of tools for developing energy-efficient concurrent applications. Moreover, we are interested here in tackling these problems under the functional programming point-of-view, using Haskell as our platform. To achieve this goal, we investigate the following key research questions:

\begin{enumerate}[label=\RQ{\arabic*}.]
  \item How can we effectively measure the energy consumption of a Haskell application?
  \item Do alternative concurrent constructs have different impacts on energy consumption?
  \item How can developers improve the energy efficiency of their concurrent applications?
\end{enumerate}

To answer \RQ{1}, we studied two performance analysis tools of the Haskell ecosystem in order to understand how they could be extended to measure the energy consumption of a Haskell program. To answer \RQ{2}, we conducted an empirical study aiming to illuminate the relationship between the choices and settings of six concurrent programming constructs available in Haskell and energy consumption. From these six, we have three thread management constructs (\forkIO, \forkOn, and \forkOS) and three primitives for sharing data (\MVar, \TMVar, and \TVar). This study consisted in an extensive experimental space exploration over both microbenchmarks and real-world Haskell programs, which were manually modified to use each of constructs that we described. Finally, to answer \RQ{3}, we elaborated a list of guidelines based on the results of our empirical study for educating developers on how to improve the energy consumption of their applications.


\section{Contributions}
This work sheds light on the energy behavior of Haskell programs. To the best of our knowledge, this is the first attempt to analyze energy efficiency in the context of functional programming languages. Moreover, this work makes the following contributions:

\begin{itemize}
  \item \textbf{A tool for fine-grained energy analysis.} We extend the \acs{ghc} profiler to collect and report fine-grained information about the energy consumption of a Haskell program;
  \item \textbf{A tool for coarse-grained energy analysis.} We extend the Criterion microbenchmarking library to collect, perform and report statistical performance analysis of the energy consumption of Haskell code;
  \item \textbf{An understanding of the energy behavior of concurrent Haskell programs.} We conduct an extensive experimental space exploration illuminating the relationship between the choices and settings of Haskell's concurrent programming constructs, and performance and energy consumption over real-world Haskell benchmarks;
  \item \textbf{A list of guidelines on how to write energy-efficient software in Haskell.} We provide some recommendations for helping software developers to write energy-efficient Haskell programs.
\end{itemize}

\section{Outline}
The remainder of this work is structured as follows.

\textbf{\chapref{chp:background}} reviews essential concepts used throughout this work. First, we briefly introduce the Haskell programming language. We show through a series of code samples some important and distinct features of the language. Second, we present an overview of the fundamentals of concurrent programming. Finally, we present how Haskell approaches concurrent programming. We show which abstractions the language provides for both creating new threads of execution and sharing data among these threads. We also present an overview of how the Haskell's runtime system handles threading on multiprocessors.

In \textbf{\chapref{chp:tools}}, we show how to measure the energy consumption of Haskell programs. First, we explain what is \acs{rapl} and how it can be used to collect energy data. Then, we present in details two performance analysis tools of the Haskell ecosystem that we extended to also work with the energy consumption metric.

\textbf{\chapref{chp:study}} shows how different concurrent constructs impact energy consumption. We present an empirical study that we conducted considering three distinct thread management constructs and primitives for sharing data. Through an extensive experimental space exploration over real-world Haskell benchmarks, we produce a list of findings about the energy behaviors of concurrent Haskell programs, which are not always obvious.

In \textbf{\chapref{chp:guide}}, we present a list of recommendations for Haskell developers on how to write energy-efficient software. These recommendations are based on the results of our empirical study.

Finally, \textbf{\chapref{chp:related}} discusses previous research related to this work and \textbf{\chapref{chp:conclusion}} present our concluding remarks and discuss where this work might lead.

It is important to point out that the \textbf{\chapref{chp:study}} of this dissertation has been published at the main research track of the \emph{IEEE International Conference on Software Analysis, Evolution, and Reengineering (SANER'16)}. When writing this dissertation, this chapter was extended and revised. A new version of this paper including these extensions and the remainder of this dissertation is under work to be submitted to a software engineering journal.
