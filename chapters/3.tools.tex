\chapter{Measuring Energy Consumption}\label{chapter:tools}
Measuring the energy consumption of a computer system is a broad area of research. There are several ways we can accomplish this task. They can be categorized into two separate approaches: power measurement and energy estimation. The first one, power measurement, makes use of special power measurement hardware to collect power samples of the running system. These samples are often measured in watts. We can then obtain the energy (in joules) multiplying the power by the time: $E = P \times t$.
%There are several power meters currently available in the market that can be used to collect this kind of data. Depending on the manufacturer, these power meters can have different characteristics. One of the most important is sampling rate, which defines the number of samples of power the is collected per second. It can vary from 1 to 1,000 samples per second. The higher the sampling rate, more accurate the final energy measurement will be.
The second approach, energy estimation, uses software-based techniques to predict how much energy the system will consume at runtime. It collects data from the running system to be used as predictors of energy consumption. For instance, powertop\footnote{https://01.org/powertop} is a Linux tool that uses this approach. It monitors CPU states, devices drivers and kernel options to report how the active components of the system are behaving regarding power consumption.

For this work, we chose to use an energy estimation approach for measuring the energy consumption of Haskell programs. In \secref{sec:rapl}, we present more details about \acs{rapl}, which is the technique we chose. Later, in \secref{sec:profiler} and \secref{sec:criterion}, we present two different performance analysis tools of the Haskell ecosystem: the \acs{ghc} profiler and Criterion. We explain how these tools work and how we extended them also to analyze energy consumption.

\section{RAPL}\label{sec:rapl}
\ac{rapl}~\citep{david:2010} is an interface designed by Intel to enable chip-level power management. \acs{rapl} is widely supported in today's Intel architectures, including Xeon family CPUs, that targets server systems, and the popular Core i5 and i7 families, that targets domestic use. This interface enables users of such processors to monitor energy consumption and set custom
power limits. \acs{rapl} uses a software power model to estimate the energy consumption based on various hardware performance counters, temperature, leakage models and I/O models~\citep{weaver:2012}.

The interaction with \acs{rapl} is done via \acp{msr}. \acp{msr} are special control registers present in the x86 instruction set that are tipically used for debugging, monitoring performance and toggling CPU features. Such MSRs can only be accessed by the operating system. In Linux, the \texttt{msr} kernel module is responsible for exposing these registers for the OS as a file inside the CPU device directories (e.g. \texttt{/dev/cpu/0/msr}). Manipulating these registers is not a straightforward process. To do this, developers need some knowledge of system programming and familiarity with the processor instruction set to know how to interpret the reading.


\section{GHC Profiler}\label{sec:profiler}
\lipsum[1-4]


\section{Criterion}\label{sec:criterion}
\lipsum[1-4]
