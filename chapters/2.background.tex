\chapter{Background}
In this chapter we review and introduce some essential concepts used in this work. First, we give a short introduction on the Haskell Programming Language. Later we describe briefly the fundamentals of concurrency and the approaches for concurrent programming in Haskell. Finally, we discuss important concepts of software energy consumption.

\section{Haskell}
Haskell is a \emph{purely functional} programming language. By functional, we mean that functions are the building blocks of programs written in Haskell. By pure, we mean that no side-effect happens when evaluating a function. While in imperative programming languages, a program is expressed as a sequence of instructions that mutates data, a program in functional programming languages is expressed as a composition of expressions where all state is controlled by passing arguments to function calls and returning values from them. This property is called \emph{referential transparency}. It guarantees that in a given execution context, a function executed with a given argument will always produce the same result. This property makes it easier to reason about programming as it enforces that a program's behavior cannot depend on history.

Haskell is also a \emph{lazy} programming language. Lazy refers to a non-strict evaluation strategy also known as \emph{call-by-need}. This strategy delays the evaluation of an expression until its value is needed. It avoids repeated evaluations which can lead to performance improvements. This strategy also makes it possible to construct potentially infinite data structures.

\emph{Recursion} is the norm in a purely functional programming language as regular iterative loops require state mutation. To make it easier to express recursive functions, Haskell also has \emph{pattern matching}. In Code \ref{code:fact}, we can see an example of a recursive function using pattern matching. Line 1 is the base case, when the \texttt{factorial} receives zero as parameter, and Line 2 is the general case.

\begin{listing}
  \begin{minted}{haskell}
factorial :: Int -> Int
factorial 0 = 1
factorial x = x * factorial (x - 1)
  \end{minted}
  \caption{A recursive factorial function}
  \label{code:fact}
\end{listing}

Another characteristic of Haskell is that functions are values. It means that a function can receive other functions as argument, and also can be the result of a function evaluation. This feature is known as \emph{high-order functions}. It enables very popular functional patterns such as \texttt{map}, \texttt{filter} and \texttt{reduce}.

Functions in Haskell can also be \emph{polymorphic}, which means that a function can be generalized to work with multiple types instead of a single one. Other programming languages have similar features such as \emph{generics} in Java and \emph{templates} in C++. Code \ref{code:poly} shows an example of a polymorphic function that can reverse a list of any type. In this example, \texttt{t} is a \emph{parametric type}~\citep{cardelli:1985} used to bind the input type to the output type of \texttt{reverse}.

\begin{listing}
  \begin{minted}{haskell}
reverse :: [t] -> [t]
reverse l = rev l []
  where
    rev []     a = a
    rev (x:xs) a = rev xs (x:a)
  \end{minted}
  \caption{A polymorphic function to reverse a list}
  \label{code:poly}
\end{listing}

In Haskell, a developer can also extend the built-in primitive types by defining new \emph{abstract data types}. Code \ref{code:tree} shows an example defining the \texttt{Tree} data type and the function \texttt{depth} to calculate the depth of the tree. As this example shows, the parametric polymorphism also works for abstract data types. Here, \texttt{Tree} have two constructors \texttt{Node} and \texttt{Empty}, where the first one holds three elements: the value of the node, the, left and right sub-trees. Pattern matching can be used to walk through a \texttt{Tree} as shown in Lines 5 and 6.

\begin{listing}
  \begin{minted}{haskell}
data Tree t = Node t (Tree t) (Tree t)
            | Empty

depth :: Tree t -> Int
depth Empty        = 0
depth (Node _ l r) = 1 + max (depth l) (depth r)
  \end{minted}
  \caption{A data type for binary tree and a function to calculate its depth}
  \label{code:tree}
\end{listing}

There is also a concept called \emph{typeclasses} that enhances the definition of new types in Haskell. A typeclass is similar to an interface. It defines a set of functions that can be applied to a particular type. Typeclasses were designed as a way for implementing ad hoc polymorphism in Haskell. Code \ref{code:tree-eq} shows an example of instantiation of the \texttt{Eq} typeclass for the \texttt{Tree} type we defined earlier. It states that a \texttt{Tree} is comparable for equality if its contained type is also comparable for equality.

\begin{listing}
  \begin{minted}{haskell}
instance Eq a => Eq (Tree a) where
  (==) Empty Empty = True
  (==) Empty (Node _ _ _) = False
  (==) (Node _ _ _) Empty = False
  (==) (Node x xl xr) (Node y yl yr) = (x == y) && (xl == yl) && (xr == yr)
  \end{minted}
  \caption{Definition of an Eq typeclass instance for the Tree data type}
  \label{code:tree-eq}
\end{listing}

The \texttt{Monad} typeclass is particularly important for Haskell because it allow developers to emulate mutable behavior and side-effects in a purely functional manner. Its definition can be seen in Code \ref{code:monad}. The most important functions of this interface are \texttt{(>==)}, also known as \emph{bind}, and \texttt{return}, also called \emph{unit}. The first one binds the contained value of the monad \texttt{m} to the argument of its argument function, and the second wraps a value in the monad \texttt{m} and returns it. It is a common idiom to call a \emph{monad} any type that is an instance of class \texttt{Monad}.

\begin{listing}
  \begin{minted}{haskell}
class Monad m where
  (>>=)  :: m a -> (a -> m b) -> m b
  (>>)   :: m a -> m b -> m b
  return :: a -> m a
  fail   :: String -> m a
  \end{minted}
  \caption{Definition of type class Monad}
  \label{code:monad}
\end{listing}

Two monads are particularly important for this work: \texttt{IO} and \texttt{STM}. The first one defines an environment to execute input/output operations. The second defines an environment for Software Transactional Memory, which is presented in Section \ref{sec:haskell-conc}. In Code \ref{code:io}, we have an example of the use of \texttt{Monad} operators to perform I/O. For instance, the functions \texttt{putStrLn} and \texttt{getLine} from the module \texttt{System.IO} prints and reads a \texttt{String} from the standard I/O, respectively. The result type of \texttt{main} is \texttt{IO ()}, where \texttt{()} is an empty tuple value. It represents that no value is returned, analogous to the type \texttt{void} on imperative languages.

\begin{listing}
  \begin{minted}{haskell}
main :: IO ()
main = putStrLn "What is your name?" >> getLine
       >>= \name -> putStrLn ("Hey " ++ name ++ ", you rock!")
  \end{minted}
  \caption{An example using the Monad operators for the IO type}
  \label{code:io}
\end{listing}

\begin{listing}
  \begin{minted}{haskell}
main :: IO ()
main = do
  putStrLn "What is your name?"
  name <- getLine
  putStrLn ("Hey " ++ name ++ ", you rock!")
  \end{minted}
  \caption{An example using the do notation}
  \label{code:io-do}
\end{listing}

Finally, there is a notation in Haskell that makes it easier to express operation within a monad, the \emph{do-notation}. Code \ref{code:io-do} shows the same example of Code \ref{code:io} rewritten using do-notation. Using \texttt{do}, a series of monadic function calls is sequenced as if in an imperative program. It works mainly as syntactic sugar for \texttt{(>==)} and \texttt{(>>)} calls, binding variables that later become argument of other functions and mainly sequentially composing the calls.


\section{Concurrency}
Concurrency and concurrent programming are on the rise nowadays due to the proliferation of multicore processors. However, these concepts are present in computer science since the early 1970s with the introduction of \emph{time-sharing}~\citep{lea:2006}. This model enables, for example, multi-tasking, which allows users to do several things at the same time in a computer such as browsing the internet, playing music and writing a document. To make this possible, the operating system scheduler has to share the processor time between all other processes that want to use this resource. By doing so, the user feels like the programs are executing at the same time although the processor is actually executing each program in a different time slice.

Concurrency is also usually associated (sometimes indistinctly) to parallelism. Although they are similar concepts, they are not the same. Concurrency consists in logically structuring programs in distinct control flows. The execution of these control flows are interlaced to simulate simultaneity when the underlying processor is single core. On the other hand, parallelism is concerned with improving a program's performance by executing several computations in parallel, which require a multicore processor. Depending on the number of cores available, the execution of a program can be literally parallel, entirely time-shared, or a combination of both.

In operating systems, there are two well-known concurrent programming abstractions to express a control flow: \emph{processes} and \emph{threads}. A process is a self-contained execution environment that holds all the information needed to run a program. Creating a new process requires a lot of computational resources such as memory, registers and address space. Due to this fact, they are known to be \emph{heavyweight}. On the other hand, threads are the smallest concurrency unit in modern operating systems~\citep{tanenbaum:2007}. As threads are contained whithin a process, there is a low overhead associated whith creating new threads because they all share the same memory and address space. Due to this fact, threads are kwon to be \emph{lightweight}.


\section{Concurrency in Haskell}\label{sec:haskell-conc}
\lipsum[1-4]


\section{Software Energy Consumption}
\lipsum[1-4]
