Há anos eficiência energética é uma preocupação para designers de hardware e software baixo-nível. Entretanto, a rápida proliferação de dispositivos móveis alimentados por bateria combinado com o crescente movimento global em busca de sustentabilidade tem motivado desenvolvedores e pesquisadores a estudar o impacto energético de softwares de aplicação em execução. Trabalhos recentes tem estudado o efeito que fatores como obsfucação de código, refatorações em linguagem orientadas à objetos e tipos de dados tem em eficiência energética. Este trabalho tenta lançar luz sobre o comportamento energético de programas concorrentes escritos em uma linguagem puramente funcional, Haskell.

Nós conduzimos um estudo empírico para avaliar o desempenho e o comportamento energético de três diferentes abordagens para gerenciamento de threads e três primitivas para controle de concorrência usando nove diferentes benchmarks com um espaço de exploração experimental de mais de 400 configurações. Neste estudo, descobrimos que pequenas mudanças podem fazer uma grande diferença em termos de consumo de energia. Por exemplo, em um dos benchmarks, sob uma configuração específica, escolher uma primitiva de controle de concorrência (MVar) ao invés de outra (TMVar) pode acarretar em uma economia de 60\% em consumo de energia. Percebemos também que nem sempre a relação entre consumo de energia e desempenho é clara. Em alguns cenários analisados, a configuração com melhor desempenho também apresentou o pior consumo de energia.

Para ajudar desenvolvedores a entender melhor essa complexa relação, nós estendemos duas ferramentas de análise de desempenho existentes para coletar e apresentar dados sobre consumo de energia. Adicionalmente, baseado nos resultados do nosso estudo empírico, listamos um conjunto de recomendações para desenvolvedores com boas práticas de como escrever código energeticamente eficiente nesse ambiente.

\begin{keywords}
Eficiência Energética, Consumo de Energia, Haskell, Programação Concorrente, Programação Funcional, Análise de Desempenho
\end{keywords}
