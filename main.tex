\documentclass[en,twoside,onehalfspacing,msc]{risethesis}

\usepackage{colortbl}
\usepackage{color}
\usepackage[table]{xcolor}
\usepackage{microtype}
\usepackage{bibentry}
\usepackage{subfigure}
\usepackage{multirow}
\usepackage{rotating}
\usepackage{booktabs}
\usepackage{pdfpages}
\usepackage{caption}
\usepackage{lipsum}

\captionsetup[table]{position=top,justification=centering,width=.85\textwidth,labelfont=bf,font=small}
\captionsetup[lstlisting]{position=top,justification=centering,width=.85\textwidth,labelfont=bf,font=small}
\captionsetup[figure]{position=bottom,justification=centering,width=.85\textwidth,labelfont=bf,font=small}

%% Change the following pdf author attribute name to your name.
\usepackage[linkcolor=black,
            citecolor=blue,
            urlcolor=black,
            colorlinks,
            pdfpagelabels,
            pdftitle={Rise Thesis Template (ABNT)},
            pdfauthor={Rise Thesis Template (ABNT)}]{hyperref}

\address{RECIFE}

\universitypt{Universidade Federal de Pernambuco}
\universityen{Federal University of Pernambuco}

\departmentpt{Centro de Informática}
\departmenten{Informatics Center}

\programpt{Pós-graduação em Ciência da Computação}
\programen{Graduate in Computer Science}

\majorfieldpt{Ciência da Computação}
\majorfielden{Computer Science}

\title{To be defined}

\date{2016}

\author{Luís Gabriel Nunes Ferreira Lima}
\adviser{Fernando José Castor de Lima Filho}
\coadviser{João Paulo de Sousa Ferreira Fernandes}

% Macros (defines your own macros here, if needed)
\def\x{\checkmark}

\begin{document}

\frontmatter

\frontpage

\presentationpage

\begin{fichacatalografica}
	\FakeFichaCatalografica % Comment this line when you have the correct file
%     \includepdf{fig_ficha_catalografica.pdf} % Uncomment this
\end{fichacatalografica}

\banca

\begin{dedicatory}
I dedicate this thesis to all my family, friends and professors who gave me the
necessary support to get here.
\end{dedicatory}

\acknowledgements
This work would not have been possible without the support of many. I would like to thank and dedicate this dissertation to the following people:

To my advisor Fernando Castor. Castor is an exceptional professor. His guidance and enthusiasm were fundamental to motivate me throughout this research.

To my co-advisor João Paulo Fernandes for his pivotal contributions to the work we have done.

To Paulo Lieuthier, Francisco Soares-Neto, and Gilberto Melfe for their numerous direct contributions to this dissertation.

To my friends, for being a part of who I am today. Special thanks go to Lucas, Tiago, Cláudio, Gustavo, Marcela, and Marina.

To my friends and colleagues from INDT. Thank you for the thought-provoking discussions, endless laughs, and beers.

To my family, especially my parents and my sister, for always supporting me through my whole life. I would not be here without them.


\begin{epigraph}[]{Poul Anderson}
I have yet to see any problem, however complicated, which, when looked at in the
right way, did not become still more complicated.
\end{epigraph}

\resumo
% Escreva seu resumo no arquivo resumo.tex
\noindent
Há anos eficiência energética é uma preocupação para designers de hardware e software baixo-nível. Entretanto, a rápida proliferação de dispositivos móveis alimentados por bateria combinado com o crescente movimento global em busca de sustentabilidade tem motivado desenvolvedores e pesquisadores a estudar o impacto energético de softwares de aplicação em execução. Trabalhos recentes tem estudado o efeito que fatores como obsfucação de código, refatorações em linguagem orientadas à objetos e tipos de dados tem em eficiência energética. Este trabalho tenta lançar luz sobre o comportamento energético de programas concorrentes escritos em uma linguagem puramente funcional, Haskell. %
Nós conduzimos um estudo empírico para avaliar o desempenho e o comportamento energético de três diferentes abordagens para gerenciamento de threads e três primitivas para controle de concorrência usando nove diferentes benchmarks com um espaço de exploração experimental de mais de 400 configurações. Neste estudo, descobrimos que pequenas mudanças podem fazer uma grande diferença em termos de consumo de energia. Por exemplo, em um dos benchmarks, sob uma configuração específica, escolher uma primitiva de controle de concorrência (MVar) ao invés de outra (TMVar) pode acarretar em uma economia de 60\% em consumo de energia. Percebemos também que nem sempre a relação entre consumo de energia e desempenho é clara. Em alguns cenários analisados, a configuração com melhor desempenho também apresentou o pior consumo de energia. %
Para ajudar desenvolvedores a entender melhor essa complexa relação, nós estendemos duas ferramentas de análise de desempenho existentes para coletar e apresentar dados sobre consumo de energia. Adicionalmente, baseado nos resultados do nosso estudo empírico, listamos um conjunto de recomendações para desenvolvedores com boas práticas de como escrever código energeticamente eficiente nesse ambiente.

\begin{keywords}
Eficiência Energética. Consumo de Energia. Haskell. Programação Concorrente. Programação Funcional. Análise de Desempenho.
\end{keywords}


\abstract
% Write your abstract in a file called abstract.tex
\lipsum[1-4]

\begin{keywords}
Software Engineering, Software Maintenance and Evolution, Change Request
Management, Automatic Change Request Assignment
\end{keywords}


% List of figures
\listoffigures

% List of tables
\listoftables

% List of acronyms
% Acronyms manual: http://linorg.usp.br/CTAN/macros/latex/contrib/acronym/acronym.pdf
\listofacronyms
\begin{acronym}[ACRONYM]
% Change the word ACRONYM above to change the acronym column width.
% The column width is equals to the width of the word that you put.
% Read the manual about acronym package for more examples:
%   http://linorg.usp.br/CTAN/macros/latex/contrib/acronym/acronym.pdf
\acro{csp}[CSP]{Communicating Sequential Processes}
\acro{tm}[TM]{Transactional Memory}
\acro{stm}[STM]{Software Transactional Memory}
\acro{ghc}[GHC]{Glasgow Haskell Compiler}
\acro{rts}[RTS]{GHC runtime system}
\acro{hec}[HEC]{Haskell Execution Context}
\acro{api}[API]{Application Programming Interface}
\end{acronym}


% Summary (tables of contents)
\tableofcontents

\mainmatter

\include{chapters/introduction}
\include{chapters/background}
\include{chapters/desenv}
\include{chapters/conclusion}

% References

\begin{references}
  \bibliography{references}
\end{references}

% Appendix

\theappendix
\include{appendix/mapping-study}

\end{document}
